\documentclass[12pt,a4paper]{article}
\usepackage{longtable}
\usepackage[landscape]{geometry}

\pdfoutput=1

\newcommand{\ek}{Simrad EK80}

\newcommand{\timesym}{t}
\newcommand{\freqsym}{f}
\newcommand{\samplesymt}{n}
\newcommand{\samplesymf}{m}
\newcommand{\genidxsym}{i}

\newcommand{\channelsym}{u}
\newcommand{\nchannels}{N_{\textrm{u}}}

\newcommand{\stagesym}{v}
\newcommand{\nstages}{N_{\textrm{v}}}

\newcommand{\fs}{f_{\textrm{s}}}
\newcommand{\fsdec}{f_{\textrm{s,dec}}}
\newcommand{\fstart}{f_{\textrm{start}}}
\newcommand{\fstop}{f_{\textrm{stop}}}
\newcommand{\fc}{f_{\textrm{c}}}
\newcommand{\fn}{f_{\textrm{n}}}


\newcommand{\zrxe}{z_{\textrm{rx,e}}}
\newcommand{\ztde}{z_{\textrm{td,e}}}

\newcommand{\ptxe}{p_{\textrm{tx,e}}}
\newcommand{\prxe}{p_{\textrm{rx,e}}}

\newcommand{\ntd}{n_{\textrm{td}}}
\newcommand{\tnom}{\tau}
\newcommand{\teff}{\tau_{\textrm{eff}}}


\newcommand{\ytxe}{y_{\textrm{tx,e}}}
\newcommand{\ytxa}{y_{\textrm{tx,a}}}
\newcommand{\yrxa}{y_{\textrm{rx,a}}}
\newcommand{\yrxe}{y_{\textrm{rx,e}}}

\newcommand{\ytx}{y_{\textrm{tx}}}
\newcommand{\ytxnorm}{\tilde{y}_{\textrm{tx}}}
\newcommand{\ytxfd}{y_{\textrm{tx,f,d}}}
\newcommand{\yrx}{y_{\textrm{rx}}}
\newcommand{\yrxorg}{y_{\textrm{rx,org}}}
\newcommand{\ymf}{y_{\textrm{mf}}}

\newcommand{\ytd}{y_{\textrm{td}}}

\newcommand{\ypc}{y_{\textrm{pc}}}
\newcommand{\ypctarget}{y_{\textrm{pc,t}}}
\newcommand{\ypcspread}{y_{\textrm{pc,s}}}
\newcommand{\ymfauto}{y_{\textrm{mf,auto}}}
\newcommand{\ymfautored}{y_{\textrm{mf,auto,red}}}

\newcommand{\ptxauto}{p_{\textrm{tx,auto}}}

\newcommand{\ypctargetf}{Y_{\textrm{pc,t}}}
\newcommand{\ypctargetnormf}{\tilde{Y}_{\textrm{pc,t}}}
\newcommand{\ypcvolumef}{Y_{\textrm{pc,v}}}
\newcommand{\ypcvolumenormf}{\tilde{Y}_{\textrm{pc,v}}}

\newcommand{\ymfautof}{Y_{\textrm{mf,auto}}}
\newcommand{\ymfautoredf}{Y_{\textrm{mf,auto,red}}}

\newcommand{\prxetf}{P_{\textrm{rx,e,t}}}
\newcommand{\prxevf}{P_{\textrm{rx,e,v}}}


\newcommand{\fstage}{k}
\newcommand{\decfac}{D}
\newcommand{\lfl}{L_{\textrm{fl}}}
\newcommand{\hfl}{h_{\textrm{fl}}}
\newcommand{\hannw}{w}
\newcommand{\hannwnorm}{\tilde{\hannw}}
\newcommand{\nw}{N_{\hannw}}
\newcommand{\hannwpart}{\gamma}
\newcommand{\tslide}{t_w} 

\newcommand{\bs}{\sigma_{\textrm{bs}}}
\newcommand{\mysp}{S_p}
\newcommand{\ts}{\textrm{TS}}
\newcommand{\sv}{S_{\textrm{v}}}

\newcommand{\range}{r}
\newcommand{\rangeref}{r_0}
\newcommand{\athw}{\phi}
\newcommand{\along}{\theta}
\newcommand{\gain}{g}
\newcommand{\gainzero}{g_0}
\newcommand{\eqang}{\psi}

\newcommand{\rtarget}{r_{\textrm{target}}}
\newcommand{\alongtarget}{\theta_{\textrm{target}}}
\newcommand{\athwtarget}{\phi_{\textrm{target}}}


\newcommand{\wlen}{\lambda}
\newcommand{\cw}{c}
\newcommand{\absorp}{\alpha}

\newcommand{\dft}{\textrm{DFT}}
\newcommand{\ndft}{{N_{\textrm{DFT}}}}
\newcommand{\ndftw}{_{\nw}}
\newcommand{\atan}{\textrm{arctan2}}
\newcommand{\anglefalong}{\gamma_\along}
\newcommand{\anglefathw}{\gamma_\athw}

\newcommand{\sigmabs}{\sigma_{\textrm{bs}}}

\newcommand{\code}[1]{\texttt{#1}} 


\renewcommand{\code}[1]{\texttt{\detokenize{#1}}}

\begin{document}

\section{List of symbols}


(1 indicates a dimensionless quantity and -- indicates not applicable).

\begin{longtable}{p{0.15\linewidth} p{0.20\linewidth} p{0.12\linewidth} p{0.5\linewidth} }
Symbol & Variable name & Units & Description \\

$\atan(y,x)$ & -- & -- & The four quadrant inverse tangent which returns values in the interval $[-\pi,\pi]$ , inclusive.\\
$\cw$ & & $m s^{-1}$ & Sound speed in water.\\
$\decfac(\stagesym)$ & & 1 & Integer filter decimation factor.\\
$\dft_\ndft(x)$ & & -- & The discrete Fourier transform of length $\ndft$ of $x$.\\
$\freqsym$ & & Hz & Frequency.\\
$\fc$ & & Hz & Frequency at centre of linear upsweep chirp pulse. Equivalent to $\fstart + \frac{|\fstop - \fstart|}{2}$.\\
$\fn$ & & Hz & Nominal operating frequency of a transducer.\\
$\fs$ & & Hz & Analogue to digital sampling frequency.\\
$\fsdec$ & & Hz & Decimated sampling frequency.\\
$\fstart$ & & Hz & Start frequency of linear upsweep chirp pulse.\\
$\fstop$ & & Hz & Stop frequency of linear upsweep chirp pulse.\\

$\gain(\along,\athw,\freqsym)$ & & 1 & Transducer gain as a function of echo arrival angle in the beam and frequency.\\

% $\gain(\along,\athw,\samplesymf)$ & & 1 & Transducer gain as a function of echo arrival angle in the beam and frequency index.\\

$\gainzero(\freqsym)$ & & 1 & Transducer gain along the main acoustic axis, i.e. $\gain(\along, \athw,\freqsym)$ where $\along = \athw = 0$.\\

$\hfl(\genidxsym,\stagesym)$ & & 1 & Complex valued receiving filter coefficients.\\
$\genidxsym$ & & 1 & Generic integer index.\\
$\Im(x)$ & & -- & The imaginary part of $x$.\\
$j$ & & & The square root of $-1$.\\
$\samplesymf$ & & 1 & Sample index in frequency domain.\\

$\nchannels$ & & 1 & Total number of transducer sectors/transceiver channels that are used to receive and process the acoustic signal.\\
$\nstages$ & & 1 & Total number of filter stages.\\
$\nw$ & & 1 & Number of samples used in the sliding Hanning window.\\
$\samplesymt$ & & 1 & Sample index in time domain.\\

$\prxetf(\freqsym)$ & & W & FT of the received electric power in a matched load for the signal from a single target at frequency $\freqsym$.\\
$\prxetf(\samplesymf)$ & & W & DFT of the received electric power in a matched load for the signal from a single target.\\
$\prxevf(\freqsym)$ & & W & FT of the received electric power in a matched load for the signal from a volume at frequency $\freqsym$.\\
$\prxevf(\samplesymf)$ & & W$\textrm{m}^2$ & DFT of the received electric power in a matched load for the signal from a volume.\\
$\prxe(\samplesymt)$  & & W & Received electric power in a matched load.\\

$\ptxe$ & & W & Transmitted electric power.\\
$\ptxauto(\samplesymt)$ & & 1 & Square of the absolute value of the matched filter autocorrelation function.\\

$\Re(x)$ & & -- & The real part of $x$.\\
$\range$ & & m & Distance from transducer.\\
$\rangeref$ & & m & Reference distance. \\
$\range_c$ & & m & Distance from the transducer to the centre of the range volume covered by $\tslide$.\\
$\range_c(\samplesymt)$ & & m & $range_c$ at sample number $\samplesymt$.\\
$\range(\samplesymt)$ & & m & Distance from transducer.\\

$\mysp(\samplesymt)$  & & dB re 1 $\textrm{m}^2$ & Point scattering strength.\\
$\sv$  & & dB re 1 $\textrm{m}^{-1}$ & Volume backscattering strength.\\
$\sv(\freqsym)$ & & dB re 1 $\textrm{m}^{-1}$ & Volume backscattering strength at frequency $\freqsym$.\\
$\sv(\samplesymt)$ & & dB re $\textrm{m}^{-1}$ & Volume backscattering strength at sample index $\samplesymt$.\\

$\ts$ & & dB re 1 $\textrm{m}^2$ & Target strength.\\
$\ts(\freqsym)$ & & dB re 1 $\textrm{m}^2$ & Target strength at frequency $f$.\\
$\ts(\samplesymf)$ & \code{TS_m} & dB re 1 $\textrm{m}^{2}$ & Target strength at frequency index $\samplesymf$.\\

$\timesym$ & & s &  Time.\\
$\tslide$ & & s & Duration of sliding window for calculating volume backscattering strength.\\

$\channelsym$ & & 1 & Channel number/transducer sector.\\
$u(\genidxsym)$ & & 1 & The Heaviside step function. \\
$V$ & & $\textrm{m}^3$ & Volume occupied by scattering targets.\\
$\stagesym$ & & 1 & Filter stage.\\

$\hannw$ & & 1 & The Hanning window function.\\
$\hannw(\genidxsym)$ & & 1 & The Hanning window function for index i, defined by $w(i) = 0.5(1+\cos (2\pi i /N_w)), -N_w/2 \leq i \leq N_w/2$.\\
$\hannwnorm(\genidxsym)$ & & 1 & Normalised Hanning window.\\

$x^*$ & & -- & The complex conjugate of $x$.\\
$||x||$ & & -- & The $l^2$-norm of $x$, also known as the Euclidean norm.\\

$\ymfautof(\samplesymf)$ & & 1 & Discrete Fourier transform (DFT) of the autocorrelation function for the matched filter.\\
$\ymfautoredf(\samplesymf)$ &  & 1 & DFT of the reduced autocorrelation function for the matched filter.\\
$\ypctargetf(\samplesymf)$ & & V & DFT of the pulse compressed signal from a single target.\\
$\ypctargetnormf(\samplesymf)$ & & Vm & DFT of the pulse compressed signal from a single target 
normalized by the DFT of the reduced autocorrelation function for the matched filter.\\
$\ypcvolumef(\samplesymf)$ & & Vm & DFT of the pulse compressed signal from a volume. Compensated for spreading loss.\\
$\ypcvolumenormf(\samplesymf)$ & & V & DFT of the pulse compressed signal from a single volume normalized by the DFT of the reduced autocorrelation function for the matched filter.  Compensated for spreading loss.\\

$\ymf(\samplesymt)$ & & 1 & Matched filter. Signal used for pulse compression.\\
$\ymfauto(\samplesymt)$ & & 1 & Autocorrelation function for the matched filter.\\
$\ymfautored(\samplesymt)$ & \code{y_mf_auto_red_n} & 1 & Reduced autocorrelation function for the matched filter.\\

$\ypc(\samplesymt)$ & & V & Pulse compressed signal averaged over all transducer sectors.\\
$\ypc(\samplesymt, \channelsym) $ & & V & Pulse compressed signal from channel $\channelsym$.\\
$y_{\textrm{pc,aft}}(\samplesymt)$ & & V & Pulse compressed signal from the aft transducer half.\\
$y_{\textrm{pc,fore}}(\samplesymt)$ & & V & Pulse compressed signal from the forward transducer half.\\
$y_{\textrm{pc,port}}(\samplesymt)$ & & V & Pulse compressed signal from the port transducer half.\\
$\ypcspread(\samplesymt)$ & & Vm & Pulse compressed signal compensated for spherical spreading.\\
$y_{\textrm{pc,star}}(\samplesymt)$ & & V & Pulse compressed signal from the starboard transducer half.\\
$\ypctarget(\samplesymt)$ & & V & Pulse compressed signal from a single target.\\

$\yrx(\samplesymt,\channelsym)$ & & V & Received digitised, bandpass filtered, decimated complex signal after the final filter stage, $\yrx(\samplesymt,\channelsym) = \yrx(\samplesymt,\channelsym,\nstages)$.\\
$\yrx(\samplesymt,\channelsym,\stagesym)$ & & V & Received digitised, bandpass filtered, decimated complex signal.\\
$\yrxa(\timesym)$ & & Pa & Analogue acoustic signal received by the transducer.\\
% $\yrxa(\timesym,\channelsym)$ & & Pa & Analogue acoustic signal received by each transducer sector $\channelsym$.\\
$\yrxe(\timesym,\channelsym)$ & & V & Analogue electric signal received by each transceiver channel $\channelsym$.\\

$\yrxorg(\samplesymt,\channelsym)$ & & V & Received digitised signal before the bandpass filtering and decimation stages, $\yrxorg(\samplesymt,\channelsym) = \yrx(\samplesymt,\channelsym,0)$.\\

$\ytx(\samplesymt)$ & & V & Ideal transmitted signal generated from transmit signal properties.\\
$\ytxnorm(\samplesymt)$ & & 1 & Ideal normalized transmitted signal generated from transmit signal properties.\\
$\ytxnorm(\samplesymt,\stagesym)$ & & 1 & Ideal normalized transmitted signal generated from transmit signal properties after application of filter stage $\stagesym$.\\

$\ytxa(\timesym)$ & & Pa & Analogue acoustic transmit signal.\\

$\ytxe(\timesym)$ & & V & Analogue electric transmit signal.\\

$y_\along(\samplesymt)$ & & rad & Electrical angle along the minor axis of the transducer (alongship when ship-mounted).\\
$y_\athw(\samplesymt)$ & & rad & Electrical angle along the major axis of the transducer (athwartship when ship-mounted).\\

$\zrxe$ & & $\Omega$ & Receiver electric impedance.\\
$\ztde$ & & $\Omega$ & Transducer sector electric impedance.\\

$\absorp(\freqsym)$  & & dB $\textrm{m}^{-1}$ & Absorption coefficient at frequency $f$.\\
% $\absorp(\samplesymf)$ & & dB $\textrm{m}^{-1}$ & Absorption coefficient at frequency index $m$\\
$\anglefalong$ & & 1 & Conversion factor between phase difference in signals from the fore and aft transducer halves and the physical arrival angle of an echo.\\
$\anglefathw$ & & 1 & Conversion factor between phase difference in signals from the port and starboard transducer halves and the physical arrival angle of an echo.\\
$\along$ & & rad & Angle coordinates along the minor axis of the transducer (alongship when ship-mounted).\\
% $\along(\samplesymt)$ & & rad & Angle of arrival of target echo along the minor axis of the transducer (alongship when ship-mounted).\\
$\wlen$ & & m &  Acoustic wavelength.\\
$\wlen_\samplesymf$ & & m & Acoustic wavelength at frequency index $m$\\
$\sigmabs$ & & $\textrm{m}^2$ & Backscattering cross-section.\\
$\tnom$ & & s & Nominal transmit pulse duration.\\
$\teff$ & & s & Effective transmit pulse duration.\\
$\athw$ & & rad & Angle coordinates along the major axis of the transducer (athwartship when ship-mounted).\\
% $\athw(\samplesymt)$ & & rad & Angle of arrival of target echo along the major axis of the transducer (athwartship when ship-mounted).\\
$\eqang(\freqsym)$ & & sr & Two-way equivalent beam angle at frequency $\freqsym$.\\
\end{longtable}

\end{document}

